\documentclass[12pt]{article}

\usepackage[margin=1in, paperwidth=8.5in, paperheight=11in]{geometry}
\usepackage{amsfonts}
\usepackage{amssymb}
\usepackage{centernot}
\usepackage{amsmath}
\usepackage{pifont}
\usepackage{xcolor}
\usepackage{mdframed}
\usepackage{graphicx}
\usepackage{fancyhdr}

\pagestyle{fancy}
\chead{CS 202 - Extra Credit 1 (10 points)}

\begin{document}

\noindent
Copy the directory \texttt{/u1/junk/cs202/ec1} to your a subdirectory of your
home directory called \texttt{ec1}. Fill in the main function of the
\texttt{csv.c} file. The program should read the \texttt{data.csv} and place
each column into its own \texttt{darray\_t} list. Then calculate the average of
each column and print them. Run the \texttt{answer} executable to see
the correctly formatted answer. You will only receive bonus points if the
answer is correct and you use the \texttt{darray\_t} struct and the \texttt{split}
function to solve the problem. Email me (Gage) when you have completed the problem and
I will collect it.\\\\
\noindent
\textbf{Note:} I realize there are easier ways to find the average of each
column; however, the point of the assignment is to dynamically read the csv
into arrays. The purpose of the average calculation is to give you something
simple to do with the columns once they have been read.

\end{document}
