\documentclass[12pt]{article}

\usepackage[margin=1in, paperwidth=8.5in, paperheight=11in]{geometry}
\usepackage{amsfonts}
\usepackage{amssymb}
\usepackage{centernot}
\usepackage{amsmath}
\usepackage{pifont}
\usepackage{xcolor}
\usepackage{minted}
\usepackage{mdframed}
\usepackage{graphicx}
\usepackage{fancyhdr}

\pagestyle{fancy}
\chead{CS 202: Assignment 4 (20 points)}

\begin{document}

\noindent
Create a subdirectory of your home directory called \texttt{a4}, and within it
write a program titled \texttt{decipher.c} that satisfies the following
requirements. The program should read a file that has been encrypted with a
substitution cipher as a command line argument, and then perform actions in the
following order:

\begin{enumerate}
  \item Read the entire file into a single buffer (using a dynamically
    allocated array).
  \item Decrypt the buffer in place using an alphabet that you guess.
  \item Count and display the frequencies of each alphabetical character.
  \item Read the words in \texttt{/u1/junk/cs202/a4/words.txt} into a dynamically
    allocated array.
  \item Using these words, count the number of English words found in the
    decrypted message and display this count.
  \item Output the decrypted message into a file called \texttt{guess.txt}.
  \item (Bonus 4 points) Decipher the file \texttt{/u1/junk/cs202/a4/secret.txt}. Using 
    the output of this program, a list of highest
    frequency letters in the English language, and by visually inspecting the
    contents of \texttt{guess.txt}, edit the alphabet your program uses until
    you believe the message has been deciphered.
\end{enumerate}

\noindent
In this program you should write five separate functions that will be used in
the main function:

\begin{enumerate}
  \item A function to read the file into a buffer.
  \item A function to decrypt the buffer in place.
  \item A function to count and print the character frequencies.
  \item A function that reads the dictionary.
  \item A function that searches the dictionary for a word via the binary
    search algorithm.
\end{enumerate}

\noindent
In addition, each function definition should have a comment above it that
describes the purpose of the function and explains the function's parameters and
return values. A simple example of this would look like:

\begin{minted}{c}

/*
    Adds two numbers together.
    Params: takes two numbers to be added together
    Return: returns the sum of the two parameters
*/
int sum(int x, int y)
{
    return x + y;
}

\end{minted}

\noindent
Run the \texttt{/u1/junk/cs202/a4/answer} executable to see how to format the
output of your program.

\end{document}
