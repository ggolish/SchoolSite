\documentclass[12pt]{article}

\usepackage[margin=1in, paperwidth=8.5in, paperheight=11in]{geometry}
\usepackage{amsfonts}
\usepackage{amssymb}
\usepackage{centernot}
\usepackage{amsmath}
\usepackage{pifont}
\usepackage{xcolor}
\usepackage{mdframed}
\usepackage{graphicx}
\usepackage{fancyhdr}

\pagestyle{fancy}
\chead{CS 202 - Quiz 1 (15 points)}

\begin{document}

\noindent
\textbf{Part 1} For each scenario listed below, list the UNIX commands you
would use to solve the problem. Number each command, starting at one.
\begin{enumerate}
  \item (2.5 points) Assume you have two directories, \texttt{dir1} and \texttt{dir2}, in
    your home directory. Change your current directory to \texttt{dir1}. Remove
    all files and subdirectories within this directory that begin with the
    letter \texttt{a}. Move all files in this directory that begin with the
    letter \texttt{b} into the \texttt{dir2} directory.
  \item (2.5 points) Create a directory in your home directory called \texttt{secret-files}.
    Copy all files in the directory \texttt{/u1/junk/cs202/dir1} that begin with
    the word \texttt{secret} into your newly created \texttt{secret-files}
    directory. Change the permissions of the directory so that only you, the
    user, have permission to read, write, and search the directory.
\end{enumerate}

\noindent
\textbf{Part 2} For each problem below, write a complete C program that
produces the desired output. Each program should print to \texttt{stdout}.
Remember to use proper spacing and indentation. The program should complain if
the input is incorrect.
\begin{enumerate}
  \item (5 points) Read an integer $x$, such that $100 \leq x \leq 10000$, as a
    command line argument. Print all prime numbers $p$, such that $0 < p
    < x$. \textit{Bonus (1 point)}: Encapsulate the code that checks for prime numbers in
    its own function.
  \item (5 points) Read one character at a time from \texttt{stdin}. If the
    character is a lowercase letter, print the corresponding uppercase letter.
    If the character is an uppercase letter, print the corresponding lowercase
    letter. If the character is not a letter, simply print the character.
    \textit{Bonus (1 point)}: If the character is a digit $n$, print the $n$th
    power of 2.
\end{enumerate}

\end{document}
