\documentclass[12pt]{article}

\usepackage[margin=1in, paperwidth=8.5in, paperheight=11in]{geometry}
\usepackage{amsfonts}
\usepackage{amssymb}
\usepackage{centernot}
\usepackage{amsmath}
\usepackage{pifont}
\usepackage{minted}
\usepackage{xcolor}
\usepackage{mdframed}
\usepackage{graphicx}
\usepackage{fancyhdr}

\pagestyle{fancy}
\chead{CS 202 - Bit Operations Exercises}

\begin{document}

\begin{enumerate}
  \item Write a function called \texttt{count1} that takes an \texttt{unsigned
    char} as an argument and returns the number of ones found in the binary
    representation of the variable.
  \item Write a function called \texttt{set} that takes two arguments: an
    \texttt{unsigned char}, $x$, and an \texttt{int}, $i$. Set
    the bit at index $i$ of $x$ to one.
  \item Write a function called \texttt{clear} that takes two arguments: an
    \texttt{unsigned char}, $x$, and an \texttt{int}, $i$. Set
    the bit at index $i$ of $x$ to zero.
  \item Write a function called \texttt{print\_bin} that takes an \texttt{int}
    as an argument and prints its binary representation to \texttt{stdout}.
  \item Write a function called \texttt{is\_odd} that takes an \texttt{int},
    $x$, as an argument. Return $1$ if $x$ is odd, and return $0$ if $x$ is
    even. Do not use the modulus operator.
  \item Write a function called \texttt{abs} that takes an
    \texttt{int}, $x$, as an argument. If $x$ is negative, perform the twos compliment operation on
    the integer and return the result. Otherwise, return $x$.
  \item Write a function called \texttt{swap\_bytes} that takes an
    \texttt{unsigned short}, $x$, as a parameter. Swap the two bytes in $x$ and
    return the result. For example: \texttt{swap\_bytes}(0x4567) returns
    0x6745.
\end{enumerate}

\end{document}
